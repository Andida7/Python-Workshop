\documentclass{beamer}
\usepackage[utf8]{inputenc}
\usepackage{graphicx}
\usepackage{subcaption}
\usepackage{listings}
%\usepackage{subfig}

\usetheme{Madrid}
\usecolortheme{seahorse}
\usefonttheme{serif} 
 
 
%Information to be included in the title page:
\title{List Data Structure}
\author{Vivek K. S., Deepak G.}
\institute{Information Systems Decision Sciences (ISDS)\\
MUMA College of Business\\
University of South Florida \\
Tampa, Florida}
\date{2017}
 
 
 
\begin{document}
 
\frame{\titlepage}
 
\begin{frame}
\frametitle{Introduction to Lists}
A list is a powerful yet simple data structure that helps us
\begin{itemize}

\item to build other complex data structures.

\item create new data types and behaviors.

\item store and retrieve data in a sequential way.

\end{itemize}

In simple terms, it is a collection of data elements ordered sequentially.

Python provides  us with a List collection type.

\end{frame}


\begin{frame}
\frametitle{The Notion of List}
\begin{itemize}

\item In the event that the language does not have a built in List type, the notion of a List has to be implemented.

\item Recall that an abstract data type is only a Mathematical notion and it needs to be physically implemented to become a data structure.

\item The behavior of the list can be easily implemented by defining functions.

\end{itemize}

There are two types of Lists

\begin{itemize}

\item An Unordered List
\item An Ordered List

\end{itemize}
\end{frame}

\begin{frame}
The Unordered List

\begin{itemize}

\item In an unordered list, each item in the collection holds a relative position with respect to other items.

\item There is no emphasis on maintaining positioning of items in contiguous memory blocks.

\item Hence there needs to be a mechanism to identify the location of each item in the memory.

\item This can be realized by using a construct known as a Linked list.

\item In a Linked list, each element holds a reference to the next element in the list.	

\item This is an explicit mechanism that helps each item identify the next item in the list.

\end{itemize}
\end{frame}

\begin{frame}
\begin{itemize}

\item This helps us to maintain the relative positioning of the items which is the integral characteristic of an unordered list.

\item Thus, each item can be accessed by simply following the links.

\item The most important knowledge here is that the location of the first item in the list must be explicitly specified.

\item The first element in the list is called the head and it holds the handle to accessing the rest of the content in the list.

\end{itemize}
\end{frame}

\begin{frame}
\frametitle{The Node}
A Node is the fundamental building block of a List.
\begin{itemize}

\item Each Node contains two essential pieces of information.

\item First, it should contain the item itself which contains the data of that particular node.

\item Second, the node contains a reference to the following node in the list.

\end{itemize}
This notion of a node can be implemented using a Class construct with attributes and methods as follows, 

\begin{itemize}
\item data - contains the data stored in the item
\item next - contains a reference to the next item.
\item methods to set and get the data and the reference.
\end{itemize}
\end{frame}

\begin{frame}[fragile]
\frametitle{Python code}

Code Implementation in Python
\begin{lstlisting}[language=Python]

class Node:
	def __init__(self, init_data):
		self.data = init_data
		self.next = None
	def get_data(self):
		return self.data
	def get_next(self):
		return self.next
	def set_data(self, new_data):
		self.data = newdata
	def set_next(self, new_next):
		self.next = new_next
\end{lstlisting}

\end{frame}

\begin{frame}[fragile]
\frametitle{The Unordered List Class}
The unordered List is constructed by using a Node as the fundamental building block. Each Node represents an item in the List.

\begin{itemize}

\item As previously stated, the reference to the first item in the list is the only way to access the entire list.

\item The class definition should therefore contain an attribute to represent the first item, which we commonly refer to us "Head".

\begin{lstlisting}[language=Python]
class UnorderedList:
	def __init__(self):
		self.head = None
\end{lstlisting}
\end{itemize}
\end{frame}

\begin{frame}
\frametitle{Continued..}
Notice that in the class definition, we used the special Python type "None".

The reasons and advantages to use "None" are many,

\begin{itemize}
\item The special reference None can be used to state that the head of the list does not refer to anything as yet.

\item if the "next" attribute of a Node contains "None", it means there is  no next node in the list.
\end{itemize}
Also it is important to understand that the list class itself does not contain any node objects. 

It only contains a single reference to only the first node in the linked structure.

\end{frame}

\begin{frame}
\frametitle{Essential Operations in a List}

\begin{itemize}
\item Create List instances on the go.
\item Ability to check if the list is empty.
\item More importantly, the ability to add items to the list in an efficient manner.
\item Ability to remove items from the list.
\item Ability to search and find an item in the list.
\item To find the number of items (size) in the list.
\end{itemize}
\end{frame}

\begin{frame}
\frametitle{Implementing the Add Method}
The most important feature of a list is the ability to add elements to it. 

And Python lists are not limited by the type of data that can be added to the list. It is versatile.

It could be implemented as follows, 
\begin{itemize}

\item Since this is an unordered list, the position of the new item in the list in relation with other items is not important.

\item The item only needs to be added to the list and become part of the linked structure.

\item Thereby, we could think of ways to add the element to the easiest position possible, which happens to be the head of the list.

\item The advantage is obvious that the add operation will always be O(1).

\end{itemize}
\end{frame}

\begin{frame}
\frametitle{Implementing the Add Method..}
\begin{itemize}
\item Recall that each item in the List is a Node and the linked list provides only a single entry point to the entire structure.

\item Hence, adding the new item to the head is the best possible approach.

\end{itemize}
It works in two important steps,
\begin{itemize}

\item The "next" reference of the new item(node) will point to the current "Head" of the list.

\item The new item then replaces the existing "Head" item to become the new head of the list.

\end{itemize}

The order of these two steps is of extreme importance to protect the integrity of the structure. 
 
\end{frame}

\begin{frame}[fragile]

\begin{lstlisting}[language=Python, keywordstyle=\color{blue}]
def add(self, item):
	temp = Node(item)
	temp.set_next(self.head)
	self.head = temp
	name="vive" # this is code
	for i in range(1,10):
		print(i)
\end{lstlisting}
\end{frame}
\end{document}

