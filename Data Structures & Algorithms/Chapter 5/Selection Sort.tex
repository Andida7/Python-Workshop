\documentclass{beamer}
\usepackage[utf8]{inputenc}
\usepackage{graphicx}
\usepackage{subcaption}
\usepackage{listings}
%\usepackage{subfig}

\usetheme{Madrid}
\usecolortheme{seahorse}
\usefonttheme{serif} 
 
 
%Information to be included in the title page:
\title{Selection Sort}
\author{Vivek K. S., Deepak G.}
\institute{Information Systems Decision Sciences (ISDS)\\
MUMA College of Business\\
University of South Florida \\
Tampa, Florida}
\date{2017}
 
 
\begin{document}
\frame{\titlepage}
\begin{frame}
\frametitle{Introduction}
\begin{itemize}
\item The selection sort improves on the bubble sort by making only one exchange for every pass through the list.
\item As every pass is made, the largest number is identified and placed in the proper position.
\item This process takes n-1 passes for a list of n items.
\end{itemize}
\end{frame}


\begin{frame}[fragile]
\frametitle{Implementation in Python}
The Python implementation is as follows:
\begin{lstlisting}[language=Python]
def selection_sort(a_list):
    for fill_slot in range(len(a_list) - 1, 0, -1):
        pos_of_max = 0
        for location in range(1, fill_slot + 1):
            if a_list[location] > a_list[pos_of_max]:
                pos_of_max = location
            a_list[fill_slot],a_list[pos_of_max] = 
            a_list[pos_of_max],a_list[fill_slot]
a = [1,2,3,4,6,345,25,6,25,5,6,72,61,6,262]
selection_sort(a)
[1, 2, 3, 4, 5, 6, 6, 6, 6, 25, 25, 61, 72, 262, 345]
\end{lstlisting}
\end{frame}


\begin{frame}
\frametitle{Analysis of Selection Sort}
\begin{itemize}
\item Selection sort makes the same number of comparisons as the bubble sort.
\item It is therefore also O(n-squared).
\item But there is a definite reduction in the number of exchanges and therefore is faster than Bubble sort.
\end{itemize}
\end{frame}

\end{document}