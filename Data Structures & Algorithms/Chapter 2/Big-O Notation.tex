\documentclass{beamer}
\usepackage[utf8]{inputenc}
\usepackage{graphicx}
\usepackage{subcaption}
\usepackage{listings}
%\usepackage{subfig}

\usetheme{Madrid}
\usecolortheme{seahorse}
\usefonttheme{serif} 
 
 
%Information to be included in the title page:
\title{Big-O Notation}
\author{Vivek K. S., Deepak G.}
\institute{Information Systems Decision Sciences (ISDS)\\
MUMA College of Business\\
University of South Florida \\
Tampa, Florida}
\date{2017}
 
\expandafter\def\expandafter\insertshorttitle\expandafter{%
  \insertshorttitle\hfill%
  \insertframenumber\,/\,\inserttotalframenumber}
 
\begin{document}
 
\frame{\titlepage}
 
\begin{frame}
\frametitle{Objectives}
\begin{itemize}
\item To understand the importance of algorithm analysis in computer science.
\item To understand the usage of “Big-O” to describe execution time.
\item To understand the “Big-O” analysis of common operations on Python lists and
dictionaries.
\item To analyze certain Python data implementations.
\item To understand how to analyze and benchmark simple Python programs.

\end{itemize}
\end{frame}

\begin{frame}
\frametitle{Introduction to Algorithm Analysis}
\begin{itemize}
\item There are many ways to solve a problem in computer science. The question here is how does one approach fair against another.
\item The underlying algorithm that the approaches represent is what that creates the difference
\item An algorithm is a generic, step-by-step list of instructions for solving a problem.
\item A software program is the algorithm encoded into a programming language of choice. 
\item Algorithm analysis is concerned with the amount of computing resources your implementation or solution uses.
\end{itemize}
\end{frame}


\begin{frame}
\frametitle{Contd..}
\begin{itemize}
\item The solution that is efficient with the limited resources available is often the best solution to a problem.
\item Sometimes, we may exchange performance for ease, but that is a subject for a whole other discussion.
\item There are two different ways to analyse performance of an algorithm
\begin{itemize}
\item One way is to consider the amount of memory an algorithm requires to solve the problem. 
\item Certain algorithms have some specific memory requirements, as we will see in certain sorting algorithms.
\item The second way is to analyze the amount of time an algorithm requires to execute. This measure is sometimes referred to as the “execution time” or “running time” of the algorithm.
\end{itemize}
\end{itemize}
\end{frame}

\begin{frame}[fragile]
\frametitle{Execution Time Example}
Lets consider the case of sum of first n integers. We will compute the execution time of our solution using Python's time module.

\begin{lstlisting}[language=Python]
import time
def sum_of_n(n):
    start = time.time()
    n_sum = 0
    for i in range(1, n+1):
        n_sum = n_sum + i
    end = time.time()
    return n_sum,end-start
\end{lstlisting}
\begin{itemize}
\item The above code takes in an integer value and computes the sum of all integers upto n.
\item We are recording the start time before the computation starts and also the end time when its done.
\item The difference will give us the full execution time of our solution.
\end{itemize}

\end{frame}

\begin{frame}[fragile]
\frametitle{Execution Time Example}
Lets run the program and see how it performs.
\begin{lstlisting}[language=Python]
for i in range(5):
    print("Sum is %d required %10.7f seconds"
     % sum_of_n(10000))
Sum is 50005000 required  0.0005009 seconds
Sum is 50005000 required  0.0005014 seconds
Sum is 50005000 required  0.0005009 seconds
Sum is 50005000 required  0.0010028 seconds
Sum is 50005000 required  0.0005014 seconds
\end{lstlisting}
The results are fairly consistent for the first 10000 integers.
\end{frame}

\begin{frame}[fragile]
\frametitle{Execution Time Example}
Lets compute the sum of the first 1000000 integers using the same program and see how that performs.
\begin{lstlisting}[language=Python]
for i in range(5):
    print("Sum is %d required %10.7f seconds"
     %sum_of_n_2(1000000))

Sum is 500000500000 required  0.0782065 seconds
Sum is 500000500000 required  0.0646460 seconds
Sum is 500000500000 required  0.0611508 seconds
Sum is 500000500000 required  0.0601616 seconds
Sum is 500000500000 required  0.0656435 seconds
\end{lstlisting}
We can see the execution time has gone up 10 times or more than that for the previous.
\end{frame}

\begin{frame}
\frametitle{Intuitions and Conclusions}
\begin{itemize}
\item Intuitively we can say that the performance of the program seems to reduce especially with larger integers.
\item But that is only expected as there is more computations to be done.
\item However, this is not uniform as we could see that the same program produces different results in different execution times on different machines, different programming languages.
\item Remember that we are only concerned with the performance of the algorithm and not the Python implementation of it.
\item Therefore, execution time is not the correct parameter to analyze performance of algorithms uniformly.
\item We need to characterize performance that is independent of the program or computer being used.
\end{itemize}
\end{frame}

\begin{frame}
\frametitle{Big-O Notation}
\begin{itemize}
\item To characterize an algorithm's performance in terms of of execution time,independent other factors  such as language or machine's ability, it is important to quantify the number of operations or
steps that the algorithm will require.
\item When each step accounts for a basic unit of computation, then the execution time for an algorithm can be expressed as the number of steps required to solve the problem. 
\item But again, deciding on an appropriate basic unit of computation can be a complicated problem in itself and depends on the algorithm's implementation.
\item In some problems, it is the number of assignments done, sometimes its the number of comparisons made between two values.

\end{itemize}
\end{frame}

\begin{frame}
\frametitle{Big-O Notation}
\begin{itemize}
\item For our problem, where we calculate the sum of n numbers, we have an initial assignment in n\_sum = 0.
\item It is followed by n number of assignments for adding every other element to the n\_sum variable.
\item Now, we denote this by a function, which we call T, where T(n) = 1 + n.
\item The parameter n is called the size of the problem and we can read this is as "T(n) is the time it takes to solve a problem of size n, namely 1+n steps.”
\item Computer scientists prefer to take this analysis technique one step further. As problems become more complex, the exact number of operations is not that important as compared to the most dominant part of the T(n) function.
\item The "order of magnitude" function describes the part of T(n) that increases the fastest as value of n increases. 
\end{itemize}
\end{frame}

\begin{frame}
\frametitle{Order of Magnitude}
\begin{itemize}
\item Order of magnitude is often called Big-O notation (for “order”) and written as
O(f(n)).
\item The representation provides a useful approximation to the actual number of steps in the computation.
\item The function f(n) provides a simple representation of the dominant part of the original T(n).
\item For example, for our sum of n integers problem, as n gets larger,the constant 1 becomes less significant and we could drop it and safely say that the execution time is O(n).
\item Lets say another example takes exactly T(n) = 2n-squared + 30n + 50 steps. But as n increases, the n=squared term gets larger and at some point it becomes the most important.
\item When it becomes even large, we could perhaps ignore the coefficient 2 as well and say the function T(n) has an order of magnitude f(n) = n-squared or O(n-squared). 
\end{itemize}
\end{frame}

\begin{frame}
\frametitle{Different Approaches to Performance Evaluation}
\begin{itemize}
\item There are different measurements taken to analyze the performance of an algorithm in different cases.
\item We usually characterize the performance in terms of best case, worst case and average case performances.
\item There are situations in which an algorithm performs especially bad but this depends entirely on the data that it is working with.
\item Going by that definition, for a different data set, the algorithm might perform exceptionally well.
\item Most of the time, the performance lies in between these two areas and we call it the average case.
\end{itemize}
\end{frame}

\begin{frame}
\frametitle{Common Order of Magnitude Functions}
\begin{itemize}
\item As we study algorithms and their analysis, we come across various functions frequently.
\begin{figure}
\includegraphics[scale=0.6]{Table}
\end{figure}
\end{itemize}
\end{frame}


\begin{frame}[fragile]
\frametitle{Evaluating Performance with Example}
Lets consider this example below and see how to analyze the performance in terms of the number of assignment expressions executed.
Although the code actually doesn't solve anything, it is a convenient example to learn how evaluation is done.
\begin{lstlisting}[language=Python]
    n= 2; a = 5; b = 6; c = 10
    for i in range(n):
        for j in range(n):
            x = i * i
            y = j * j
            z = i * j
    for k in range(n):
        w = a * k + 45
        v = b * b
    d = 33
\end{lstlisting}
\end{frame}

\begin{frame}[fragile]
\frametitle{Evaluating Performance with Example}
\begin{itemize}
\item Initially we have 4 assignment statements which we will consider as constant '4' in our function.
\item In the next block of code, we see there is a nested for loop which accounts for $n^2$ steps and since there are 3 assignment statements enclosed in it, they totally account for 3$n^2$ assignments.
\item That is followed by another for loop which has 2 assignment statements and therefore, 2n steps.
\item At the end there is an assignment again and so it is 1 step and hence the final function is  
T(n) = 4 + 3$n^2$ + 2n + 1
\item But in the event, n gets larger, the 3$n^2$ part will be dominating and hence, we can approximate this as O($n^2$) algorithm.
\end{itemize}
\end{frame}

\begin{frame}[fragile]
\frametitle{Anagram Detection Example}
Lets discuss Big-O Notation further with the Anagram example.
\begin{itemize}
\item One string is an anagram of another if the second is simply a rearrangement of of the first.
\item For example, earth is a rearrangement of heart. Likewise "Mother in law" is an anagram of "Hitler woman".
\item If the two strings are of  the same length and are made up of a combination of the 26 alphabets, we could easily analyze them.
\item Lets discuss two solutions to identifying an anagram and analyze their performances.
\item Code link to be shared.
\end{itemize}
\end{frame}


\begin{frame}
\frametitle{The Analysis of the Solution 
\#1}
\begin{itemize}
\item At first glance you we see straightforward assignments and hence we might be tempted to think this O(n).
\item But the calls to the sort method comes with a price.
\item We will see in later exercises that the sort methods are either O(nlog n) or O($b^2$).
\item And for larger n values, the sorting operation will dominate the performance.
\item Therefore, this algorithm will have the same order of magnitude as that of the sorting process.
\end{itemize}
\end{frame}


\begin{frame}
\frametitle{The Analysis of the Solution 
\#2}
\begin{itemize}
\item The solution has a number of iterations. But, unlike the first solution, none of them
are nested. 
\item The first two iterations used to count the characters are both based on n.
\item The third iteration that compares the two lists of counts, always takes 26 steps since there are 26 possible characters in the strings.
\item Adding them up, it gives us T(n) = 2n + 26 steps. That is O(n).
\item We have thus achieved a linear order of magnitude algorithm for solving this problem.
\item Note - Although this solution achieved a better performance, it is important that we made some memory sacrifices to attain it.
\item The solution could only work by using additional storage to keep the two lists of character counts.
\end{itemize}
\end{frame}

\begin{frame}
\frametitle{Leaving Thoughts}
\begin{itemize}
\item In computer science, it is often left to the nature of the problem, the programmer and the final output that is expected of it, to make decisions on performance and space requirements.
\item In the example that we just discussed, although the second solution appears to be the better one, it could create serious strain on the memory if the strings to be compared were large.
\item Usually, the problems and data are quite complex and large and it is up to us find a balance between performance and resource constraints.
\end{itemize}
\end{frame}
\end{document}

