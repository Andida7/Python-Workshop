\documentclass{beamer}
\usepackage[utf8]{inputenc}
\usepackage{graphicx}
\usepackage{subcaption}
\usepackage{listings}
%\usepackage{subfig}

\usetheme{Madrid}
\usecolortheme{seahorse}
\usefonttheme{serif} 
 
 
%Information to be included in the title page:
\title{Tree Data Structure}
\author{Vivek K. S., Deepak G.}
\institute{Information Systems Decision Sciences (ISDS)\\
MUMA College of Business\\
University of South Florida \\
Tampa, Florida}
\date{2017}
 
 
 
\begin{document}
 
\frame{\titlepage}
\begin{frame}
\frametitle{Introduction}
\begin{itemize}
\item Queue is a first-in-first-out data structure and it has a very important variation called a priority queue.
\item A priority queue acts like a queue, except that the dequeueing happens from the front of the queue.
\item The logical ordering of items insider a priority queue is also determined by their priority.
\item The classic way to implement a priority queue is by using a data structure called a binary heap.
\item The binary heap is implemented with a single list as an internal representation and has two variations, the "min heap" and the "max heap".
\item In the min heap, the smallest key is at the front and its the opposite in the max key.
\end{itemize}
\end{frame}


\begin{frame}
\frametitle{Essential Operations in a Binary Heap}
\begin{itemize}
\item Ability to create a new empty Binary Heap.
\item Ability to insert a new item.
\item Finding the item with the minimum key value in the heap.
\item Removing the item with the minimum key value.
\item Find out if the heap is empty.
\item Find out the size of the heap.
\item Building a new heap from a list of keys.
\end{itemize}
\end{frame}

\begin{frame}
\frametitle{Properties of a Binary Heap}
Lets discuss the properties of a Binary heap before we go ahead with the implementation.
Structure Property:
\begin{itemize}
\item For the heap to work efficiently, it must preserve the logarithmic nature of the binary tree.
\item The tree must remain balanced, to ensure logarithmic performance. A balanced binary tree has roughly the same number of nodes in the left and right subtrees.
\item A complete binary tree is a tree in which each level has all its nodes except the bottom level of the entire tree.
\item An interesting property of a complete tree is that we can represent it using a single list.
\item In such an implementation, the node at index p is the parent of the node found at position 2p which is actually its left child.
\item Therefore, intuitively, the node at 2p+1 becomes the right child.
\end{itemize}
\end{frame}

\begin{frame}
\frametitle{Contd..}
\begin{itemize}
\item Using this property, we can easily identify the parent of any node using simple integer division.
\item For a node at position p, its parent node will be located at p//2. Remember that integer division only returns the quotient part of the division.
\item Therefore, this method works for both nodes (left or right child node).
\item Thus, in a complete binary tree, where the structural property is preserved, the nodes can be traversed easily using simple mathematical operations.
\end{itemize}
\end{frame}

\begin{frame}
\frametitle{Heap Order Property}
\begin{itemize}
\item Along with the structure property, the Heap order property is very important in building a Binary Heap.
\item Its defined as follows: For every node n with parent p, the key in p is smaller than or equal to the key in n.
\item These two properties are absolutely necessary to be in place for us.
\end{itemize}
\end{frame}

\begin{frame}[fragile]
\frametitle{Python Implementation}
\begin{itemize}
\item As discussed before, the Binary heap can be represented by a single list. The class definition will be as follows:
\item We will initialize a list and also add another attribute to keep track of the size of the list. 
\item For ease of doing mathematical operations, we will add a single element to the list (just as a placeholder). We will add a zero for this purpose.
\item Note that this zero has no significance in this implementation whatsoever. It is added just to make the math easier.
\begin{lstlisting}[language=Python]
class Binary_Heap:
    def __init__(self):
        self.h_list = [0]
        self.size = 0
\end{lstlisting}
\end{itemize}
\end{frame}

\begin{frame}[fragile]
\frametitle{Adding an Item}
\begin{itemize}
\item Insertion can be done by using the append method of lists.
\item But it is imperative that we maintain the heap order property while doing it.
\item Hence, when the item added is less than its parent, we need to move it up the tree.
\item We call this percolate up.
\item We may have to percolate the newly added item up till we find that the parents and child nodes are all in order.
\item Here is the implementation of this logic.
\begin{lstlisting}[language=Python]
def percolate_up(self, i):
    while i//2 > 0:
        if self.h_list[i] < self.h_list[i//2]:
            self.h_list[i], self.h_list[i//2] = 
            self.h_list[i//2], self.h_list[i]
        i = i//2
\end{lstlisting}
\end{itemize}
\end{frame}

\begin{frame}[fragile]
\frametitle{Contd..}
We are now ready to write the insert method. Most of the work in the insert method is really done by percolate\_up. 

Once a new item is appended to the tree, percolate\_up takes over and positions the new item properly.

\begin{lstlisting}[language=Python]
def insert(self, item):
    self.h_list.append(item)
    self.size = self.size + 1
    self.percolate_up(self.size)
\end{lstlisting}
As you can see, once the item is appended, the size is incremented by 1 and then the control goes to the percolate\_up method.
\end{frame}

\begin{frame}[fragile]
\frametitle{Delete Minimum}
Deleting the minimum key in the heap is again going to violate the heap structure and order property.
\begin{itemize}
\item Finding the min item is easy but restoring the order and structure of the heap is the hard part.
\item We'll do that in two steps. First, after removal of the minimum element, we'll move the last item in the list to the root position.
\item Moving the last item restores the structure of the heap. But we need to restore the order and for that we need to move the new root down the tree  to its appropriate position.
\item We  call this percolate down.
\item Using the method, we swap the parent and child nodes until the order is achieved.
\end{itemize}
\end{frame}

\begin{frame}[fragile]
\frametitle{Implementation}
The percolate down method
\begin{lstlisting}[language=Python]
def percolate_down(self, i):
    while (i * 2) <= self.current_size:
        mc = self.min_child(i)
        if self.h_list[i] > self.h_list[mc]:
            self.h_list[i],self.h_list[mc] =
             self.h_list[mc],self.h_list[i]
        i = mc
\end{lstlisting}

Method to identify the min child.
\begin{lstlisting}[language=Python]
def min_child(self, i):
    if i * 2 + 1 > self.size: return i * 2
    else:
        if self.h_list[i*2] < self.h_list[i*2 +1]:
            return i * 2
        else: return i * 2 + 1
\end{lstlisting}
\end{frame}

\begin{frame}[fragile]
\frametitle{Implementation}
Now that we have implemented all the supporting methods, we will implement the delete method as follows:
\begin{lstlisting}[language=Python]
def del_min(self):
    ret_val = self.h_list[1]
    self.h_list[1] = self.h_list[self.size]
    self.size = self.size - 1
    self.h_list.pop()
    self.percolate_down(1)
    return ret_val
\end{lstlisting}
\end{frame}

\begin{frame}
\frametitle{Summary}
\begin{itemize}
\item Binary Heaps is an important tree data  structure wherein the properties of a queue and a tree are combined together.
\item The properties of a binary heap dictates that the structure property and the order property have to be maintained to get the best performance from the data structure.
\item It could be implemented using a single list.
\end{itemize}
\end{frame}
\end{document}