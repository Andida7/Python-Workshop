\documentclass{beamer}
\usepackage[utf8]{inputenc}
\usepackage{graphicx}
\usepackage{subcaption}
\usepackage{listings}
%\usepackage{subfig}

\usetheme{Copenhagen}
\usecolortheme{seahorse}
 
 
%Information to be included in the title page:
\title{Introduction to Numpy}
\author{Vivek K. S., Deepak G.}
\institute{Information Systems Decision Sciences (ISDS)\\
MUMA College of Business\\
University of South Florida \\
Tampa, Florida}
\date{2017}
 

\begin{document}
\frame{\titlepage}
\begin{frame}
\frametitle{Introduction to Numpy}
\begin{itemize}
\item Numpy is a Python package used for numerical computation and multi-dimensional array operations.
\item It is a vast library of methods and modules that support a wide variety of operations.
\item It is the  fundamental building block of higher-level packages such as Pandas and TensorFlow.
\item Numpy provides built-in objects (ndarrays) which are multi-dimensional arrays of homogeneous data type.
\item It provides vectorized operations on multi-dimensional arrays which are very fast and efficient compared to iterative  operations.
\end{itemize}
\end{frame}

\begin{frame}[fragile]
\frametitle{Importing Numpy and Creating Arrays}
We can import the numpy package as follows: \\
\begin{lstlisting}[language=Python]
import numpy
\end{lstlisting}
We can create aliases for the package we download so that we could call the methods more easily as follows:
\begin{lstlisting}[language=Python]
import numpy as np
\end{lstlisting}
Creating a numpy array from a list is simple as follows:
\begin{lstlisting}[language=Python]
import numpy as np
first_array = np.array([1,2,3,4,5])
\end{lstlisting}
\end{frame}

\begin{frame}[fragile]
\frametitle{Creating Multi-dimensional Arrays}
Multi-dimensional arrays can be created as follows:
\begin{lstlisting}[language=Python]
a = [1, 2, 3, 4, 5]
b = [1,4,9,16,25]
squares = np.array([a,b])
print(squares)

[[ 1  2  3  4  5]
 [ 1  4  9 16 25]]
 
# Checking the dimensions
print(squares.shape)
(2,5)
\end{lstlisting}
\end{frame}

\begin{frame}[fragile]
\frametitle{Checking the Type of the Array}
Checking the type of the array is as easy as follows:
\begin{lstlisting}[language=Python]
squares.dtype
dtype('int32')

# Creating a 3-dimensional array
# We can use the same lists again and add an
 additional dimensional array to it.
cubes = np.array([[a,b,[1,8,27,64,225]]])
cubes.ndim
3

# Accessing the elements using indexing
cubes[0][1]
array([ 1,  4,  9, 16, 25])
\end{lstlisting}
\end{frame}

\begin{frame}[fragile]
\frametitle{Common Numpy Operations}

\begin{lstlisting}[language=Python]
a = np.array([1,2,3,4])
# Get the type of a 
type(a)
numpy.ndarray
# assigning a float to an int array will
# truncate the decimal part.
a[0] = 5.6
print(a)
array([5, 2, 3, 4])
# Convert to list
b = a.tolist()
type(b)
list

\end{lstlisting}
\end{frame}


\begin{frame}[fragile]
\frametitle{Create Numpy Array from Random}
\begin{lstlisting}[language=Python]
# Create a numpy array from a random set of
 integers with a specific size
rand_array = np.random.randint(100, size=(6, 6))

[[14 36 45 51 94 59]
 [22 76 84 16 77 36]
 [ 4 62 76 45 32 94]
 [77 22 84 56 47 82]
 [18 54 10 86 88 81]
 [86 32  2 96 82 33]]
\end{lstlisting}
\end{frame}

\begin{frame}[fragile]
\frametitle{Slice and Dice a Numpy Array}
\begin{lstlisting}[language=Python]
# get only the column 3 through 4 from row 0
rand_arr[0,3:5] 
array([51, 94])
# Get the elements at the bottom right corner
rand_arr[4:,4:]
array([[88, 81],
       [82, 33]])
# Get a complete row
rand_arr[2,:] # the 3rd row .. i.e index 2
array([ 4, 62, 76, 45, 32, 94])
# Get a complete row 
rand_arr[:,3] # the 4th row.. i.e index 3
array([51, 16, 45, 56, 86, 96])
\end{lstlisting}
\end{frame}

\begin{frame}[fragile]
\frametitle{Alternate Rows and Columns}
\begin{lstlisting}[language=Python]
rand_arr[::2] # Getting alternative rows
array([[14, 36, 45, 51, 94, 59],
       [ 4, 62, 76, 45, 32, 94],
       [18, 54, 10, 86, 88, 81]])
rand_arr[:,::2] # Getting alternate columns
array([[14, 45, 94],
       [22, 84, 77],
       [ 4, 76, 32],
       [77, 84, 47],
       [18, 10, 88],
       [86,  2, 82]])
\end{lstlisting}
\end{frame}

\begin{frame}[fragile]
\frametitle{Taking Strides in The Array}
\begin{lstlisting}[language=Python]
# Lets print random array for the next exercise
rand_arr
array([[14, 36, 45, 51, 94, 59],
       [22, 76, 84, 16, 77, 36],
       [ 4, 62, 76, 45, 32, 94],
       [77, 22, 84, 56, 47, 82],
       [18, 54, 10, 86, 88, 81],
       [86, 32,  2, 96, 82, 33]])
rand_arr[2::2,::2]
array([[ 4, 76, 32],
       [18, 10, 88]])
\end{lstlisting}
\end{frame}

\begin{frame}[fragile]
\frametitle{Slices are References}
\begin{lstlisting}[language=Python]
# Lets create an array
a = np.array([1,2,34,5,563])
print(a)
array([  1,   2,  34,   5, 563])
# Lets take a slice of a and assign it to b
b = a[2:5]
# Lets add an item to b's 0th index position
b[0] = 252
# But turns out a changed as well along with b
a
\end{lstlisting}
That is because slices are references and not separate objects and hence any change made through the references pointing to a slice of an array creates changes in the original array. This is called broadcasting.
\end{frame}

\begin{frame}
\frametitle{Summary}
\begin{itemize}
\item 
\end{itemize}
\end{frame}
\end{document}

