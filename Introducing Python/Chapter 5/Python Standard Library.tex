\documentclass{beamer}
\usepackage[utf8]{inputenc}
\usepackage{graphicx}
\usepackage{subcaption}
\usepackage{listings}
%\usepackage{subfig}
\usetheme{Copenhagen}
\usecolortheme{seahorse}
 

%Information to be included in the title page:
\title{Python Standard Library}
\author{Vivek K. S., Deepak G.}
\institute{Information Systems Decision Sciences (ISDS)\\
MUMA College of Business\\
University of South Florida \\
Tampa, Florida}
\date{2017}
 
\begin{document}
\frame{\titlepage}

\begin{frame}[fragile]
\frametitle{Python Standard Library}
\begin{itemize}
\item The Python Standard Library offers ready to import modules for most common tasks in Programming with Python.
\item These modules help a lot as they perform many useful tasks are kept away from the core language to avoid bloating.
\item Some Useful resources.
\begin{itemize}
\item Documentation of Modules - https://docs.python.org/3/library/
\item Tutorial on their usage - https://docs.python.org/3.3/tutorial/stdlib.html
\item Module of the Week by Doug
Hellmann - https://pymotw.com/2/contents.html
\item The Python Standard Library By Example Book - https://doughellmann.com/blog/the-python-standard-library-by-example/
\end{itemize}
\end{itemize}
\end{frame}

\begin{frame}[fragile]
\frametitle{Counter()}
Counter() is used to count the number of occurrences of a unique element in a sequence.
Example,
\begin{lstlisting}[language=Python]
from collections import Counter
list = ['Orange','Apple','Mango','Orange']
list_counter = Counter(list)
list_counter
Counter({'Apple': 1, 'Mango': 1, 'Orange': 2})
\end{lstlisting}
The most\_common() function returns all elements in descending order, or just the top
count elements if given a count:
\begin{lstlisting}[language=Python]
list_counter.most_common()
[('Orange', 2), ('Mango', 1), ('Apple', 1)]
\end{lstlisting}
\end{frame}

\begin{frame}
\frametitle{Deque}
Deque is a special data structure that is a combination of both Stack and Queue. A deque is a double-ended queue. 
\begin{itemize}

\item Its useful to add and delete elements from both ends of the sequence.
\item The function popleft() removes the leftmost item and pop() removes the rightmost element.
\item A deque could be used to create a palindrome checker as follows.
\end{itemize}
\end{frame}

\begin{frame}[fragile]
\frametitle{Palindrome Checker Using Deque}
\begin{lstlisting}[language=Python]
def palindrome_checker(word):
    from collections import deque
    dq = deque(word)
    while len(dq) > 1:
        if dq.popleft() != dq.pop():
            return False
    return True

palindrome_checker('racecar')
True
palindrome_checker('hello')
False
\end{lstlisting}
\end{frame}

\begin{frame}[fragile]
\frametitle{Iterate with Itertools}
Itertools contains special-purpose iteration functions that come in handy in a lot of scenarios.

chain() runs through each argument as though they are all part of one iterable:

\begin{lstlisting}[language=Python]
import itertools
for item in itertools.chain([1, 2], ['one', 'two']):
	print(item)
\end{lstlisting}

cycle() is an infinite iterator, cycling through its arguments:
\begin{lstlisting}[language=Python]
import itertools
for item in itertools.cycle([1, 2]):
    print(item)
\end{lstlisting}

\end{frame}


\begin{frame}[fragile]
\frametitle{Iterate with Itertools}

accumulate() calculates accumulated values. By default, it calculates the sum:

\begin{lstlisting}[language=Python]
import itertools
for item in itertools.accumulate([1, 2, 3, 4]):
    print(item)
\end{lstlisting}

If a function is passed as the second argument to accumulate(), it will supersede the default addition operation.
\begin{lstlisting}[language=Python]
def multiply(a, b):
    return a * b
    
import itertools
for item in itertools.accumulate([1, 2, 3, 4],
multiply):
    print(item)
\end{lstlisting}
The function should take two arguments and return a single result.
\end{frame}

\begin{frame}[fragile]
\frametitle{Print Pretty Statements with pprint()}
pprint pretty prints data for us.
\begin{lstlisting}[language=Python]
data = [ (i, { 'a':'A','b':'B','c':'C',
'd':'D','e':'E'})
         for i in range(3)
         ]
from pprint import pprint

# Check the difference between 
print(data)
pprint(data)
\end{lstlisting}
\end{frame}

\begin{frame}
\frametitle{Summary}
\begin{itemize}
\item We learned how Python pushed all the custom, advanced functionalities into its Standard library so that the language would not become too bloated.
\item We learned how to use different tools such as Counter, Deque, itertools and pprint to make our code deliver more.
\item We learned how to import these modules into our code as and when we need, even selectively, rather than importing the entire package.
\end{itemize}
\end{frame}
\end{document}
