\documentclass{beamer}
\usepackage[utf8]{inputenc}
\usepackage{graphicx}
\usepackage{subcaption}
\usepackage{listings}
%\usepackage{subfig}

\usetheme{Madrid}
\usecolortheme{seahorse}
\usefonttheme{serif} 
 
 
%Information to be included in the title page:
\title{Working With Files and Directories}
\author{Vivek K. S., Deepak G.}
\institute{Information Systems Decision Sciences (ISDS)\\
MUMA College of Business\\
University of South Florida \\
Tampa, Florida}
\date{2017}
 
\expandafter\def\expandafter\insertshorttitle\expandafter{%
\insertshorttitle\hfill%
\insertframenumber\,/\,\inserttotalframenumber}

\begin{document}
\frame{\titlepage}

\begin{frame}
\frametitle{Introduction to Files and Directories}
\begin{itemize}
\item We usually work with file and directories when we work on computers.
\item Our data is stored in the form of files in directories.
\item Often, we will need to create files in our programs to store our output data for later use.
\item Programs can be written also to modify the data stored in a file.
\item In this module, we will see how to work with files and folders in Python.
\item Python's file operations are modeled after the Unix System and is available through the OS module.
\end{itemize}
\end{frame}

\begin{frame}[fragile]
\frametitle{Common File Operations}
We can create a file by simply using the open method. If the file specified doesn't already exist, it will be created.
\begin{lstlisting}[language=Python]
import os
fout = open('my_file.txt', 'wt')
fout.close()
# Check if the file has been created with exists()
os.path.exists('my_file.txt')
True
\end{lstlisting}
\end{frame}

\begin{frame}[fragile]
\frametitle{Common File Operations}
We can check if the given path is a file or a directory by using the isfile() and isdir() methods.
\begin{lstlisting}[language=Python]
name = 'my_file.txt'
os.path.isfile(name)
True

# to check if its a directory
os.path.isdir(name)
False
\end{lstlisting}
\end{frame}


\begin{frame}[fragile]
\frametitle{Common File Operations}
Some of the other commonly used file  operations are:
\begin{lstlisting}[language=Python]
# Get the current working directory
os.getcwd()
# Get all the files in the current workind direcorry
os.listdir()
# Rename a file
os.rename('old_file_name', 'new_file_name')
\end{lstlisting}

More coding examples at - link
\end{frame}

\begin{frame}
\frametitle{Calendar and Date}
\end{frame}

\begin{frame}
\frametitle{Summary}
\begin{itemize}
\item We learned how to use Python's os module to perform different file and directory related operations.
\item We learned how to use Python to interact with the operating system to accomplish system tasks.
\item We learned how to deal with Calendar, Date and Time in Python.
\end{itemize}
\end{frame}
\end{document}

