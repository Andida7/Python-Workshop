\documentclass{beamer}
\usepackage[utf8]{inputenc}
\usepackage{graphicx}
\usepackage{subcaption}
\usepackage{listings}
%\usepackage{subfig}
\usetheme{Copenhagen}
\usecolortheme{seahorse}
 
 
%Information to be included in the title page:
\title{Dictionaries and Sets}
\author{Vivek K. S., Deepak G.}
\institute{Information Systems Decision Sciences (ISDS)\\
MUMA College of Business\\
University of South Florida \\
Tampa, Florida}
\date{2017}
 
\begin{document}
\frame{\titlepage}

\begin{frame}
\frametitle{Dictionaries}
\begin{itemize}
\item A dictionary is similar to a list, in that it is a collection of items.
\item Unlike lists, the order of the items doesn't matter.
\item Dictionary elements are not identified by offset indicing as seen in lists (0,1,2..).
\item Instead, in dictionaries, the items are key-value pairs and they are uniquely identified by the keys as they are unique. 
\item The keys are usually of String type, but they can be any immutable type in Python as integers, Booleans, floats and Tuples.
\item Dictionaries are perfectly mutable.
\end{itemize}
\end{frame}

\begin{frame}[fragile]
\frametitle{Creating a Dictionary}
\begin{lstlisting}
>>>empty_dict = {}
>>>empty_dict
{}
# Creating a dictionary with key-value pairs.
>>>capitals={'New York':'Albany',
'California':'Sacramento'}
>>>capitals
{'California': 'Sacramento', 'New York': 'Albany'}
\end{lstlisting}
\end{frame}

\begin{frame}[fragile]
\frametitle{Converting other Sequences into Dictionaries}
\begin{lstlisting}
# A list of two-item tuples:
>>>my_tuple = [ ('a', 'b'), ('c', 'd'), ('e', 'f') ]
>>>my_dict = dict(my_tuple)
{'a': 'b', 'c': 'd', 'e': 'f'}

# A list of two-item lists:
>>>my_list = ( ['a', 'b'], ['c', 'd'], ['e', 'f'] )
>>>my_dict = dict(my_list)
{'c': 'd', 'a': 'b', 'e': 'f'}
\end{lstlisting}
\end{frame}

\begin{frame}[fragile]
\frametitle{Continued..}
\begin{lstlisting}
# A list of two-character strings:
>>>my_str_list = [ 'ab', 'cd', 'ef' ]
>>>my_dict = dict(my_str_list)
>>>my_dict
{'a': 'b', 'c': 'd', 'e': 'f'}

# A tuple of two-character strings:
>>>my_str_tuple = ( 'ab', 'cd', 'ef' )
>>>my_dict = dict(my_str_tuple)
>>>my_dict
{'a': 'b', 'c': 'd', 'e': 'f'}
\end{lstlisting}
\end{frame}

\begin{frame}
\frametitle{Operations in Dictionaries}
\begin{itemize}
\item Adding and modifying key-value pairs.
\item Combine Dictionaries with Update().
\item Deleting items.
\item Deleting all items using clear().
\item Membership test.
\item Fetching items from the dictionary.
\item Assigning using assignment sign and by copying.
\end{itemize}
\end{frame}

\begin{frame}[fragile]
\frametitle{Sets in Python}
\begin{itemize}
\item Sets are similar to lists, but only the items are unique. No duplication is allowed.
\item A set can be created as follows: set() or by using the curly braces with one or more comma-separated values.
\item Sets are unordered just like Dictionaries.
\item An empty pair of curly braces will only create a dictionary and not a set.
\item The Python interpreter prints an empty set as set() and not as {}.
\end{itemize}
\end{frame}
\end{document}