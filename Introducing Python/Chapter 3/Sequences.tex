\documentclass{beamer}
\usepackage[utf8]{inputenc}
\usepackage{graphicx}
\usepackage{subcaption}
\usepackage{listings}
%\usepackage{subfig}
\usetheme{Copenhagen}
\usecolortheme{seahorse}
 
 
%Information to be included in the title page:
\title{Python Sequences}
\author{Vivek K. S., Deepak G.}
\institute{Information Systems Decision Sciences (ISDS)\\
MUMA College of Business\\
University of South Florida \\
Tampa, Florida}
\date{2017}

\expandafter\def\expandafter\insertshorttitle\expandafter{%
\insertshorttitle\hfill%
\insertframenumber\,/\,\inserttotalframenumber}

\lstset{language=python,
		showstringspaces=false,
                basicstyle=\ttfamily,
                keywordstyle=\color{blue}\ttfamily,
                stringstyle=\color{red}\ttfamily,
                commentstyle=\color{purple}\ttfamily,
                morecomment=[l][\color{magenta}]{\#}
}
 
\begin{document}
\frame{\titlepage}
\begin{frame}
\frametitle{Objectives}
\begin{itemize}
\item To understand built-in data sequence types such as Lists, Sets, Tuples and Dictionaries.
\item To understand their applications and advantages.
\item To understand iterators and constructs that support sequences.
\item To learn the different operations that could be performed on these data types.
\end{itemize}
\end{frame}
\begin{frame}
\frametitle{Lists}
\begin{itemize}
\item A list is a collection of data elements arranged sequentially.
\item They are mutable in nature, and data can inserted removed and appended.
\item Lists are enclosed by a pair of squared brackets '[]'
\item Lists in Python are \textbf{heterogeneous}. A list can contain any different type of data.
\item Lists are especially good for keeping track of things that could change in the course of the program execution.
\item A list could be created by a pair of squared-brackets [] or list().
\end{itemize}
\end{frame}


\begin{frame}[fragile]
\frametitle{Creating a List}
A list could be created as follows:
\begin{lstlisting}[language=Python]
cities = ['Chicago','San Francisco','New York']
# A list could also become an 
# element of another list 
metros = ['Dallas','Las Vegas', cities]
print(cities)
['Chicago', 'San Francisco', 'New York']
print(metros)
['Dallas', 'Las Vegas', ['Chicago', 
'San Francisco', 'New York']]
\end{lstlisting}
\begin{itemize}
\item Changing the cities list here modifies the  metros list as well.
\item This is because, cities is just a reference to the block of memory containing "cities list" and changes to it is being reflected in the metros list.
\end{itemize}
\end{frame}

\begin{frame}[fragile]
\frametitle{Data Type Conversions in Lists}
Converting a string to a list is easy as follows:
\begin{lstlisting}[language=Python]
language = list('Python')
print(language)
['P', 'y', 't', 'h', 'o', 'n']

# Converting String to List using split()
birthday = '8/11/2017'
dob = birthday.split('/')
print(dob)
['8', '11', '2017']

# List can be converted to a String using join() as follows:
statement = ['I','love','Python']
sentence = ' '.join(statement)
print(sentence)
'I love Python'
\end{lstlisting}
\end{frame}

\begin{frame}
\frametitle{Common List Operations}
The following are the common operations in List.
\begin{itemize}
\item Adding elements to the list using add, append and insert.
\item Indexing and Slicing for data retrieval.
\item Modifying the list using indexing and Slicing techniques.
\item Concatenating Lists.
\item Removing items from the list using "del" remove and pop.
\item Determining the membership of an item and counting the number of items.
\item Searching and Sorting the list.

\end{itemize}
\end{frame}

\begin{frame}
\frametitle{More Examples in Lists}
More code examples are available at --
 
\url{https://github.com/vivek14632/Python-Workshop/tree/master/Introducing\%20Python/Chapter\%203}
\end{frame}

\begin{frame}
\frametitle{Tuples}
Similar to lists, tuples are sequences of arbitrary items. 
\begin{itemize}
\item Tuples are the immutable collection sequence types in Python.
\item Tuples can’t be modified using add, delete, or changing of items once the tuple is created. A tuple is thus a constant list type.
\item A tuple can be created using a pair of parenthesis ().
\item Notice that it is different from how a list can be created by saying "list()".
\end{itemize}
\end{frame}

\begin{frame}
\frametitle{Tuples}
Often, there is a debate over how the word Tuple is to be pronounced. Guido put an end to this debate in one of his Tweets as follows:

\textit{"I pronounce tuple too-pull on Mon/Wed/Fri and tub-pull on Tue/Thu/Sat. On Sunday I don't talk about them. :) @avivby"}
\end{frame}

\begin{frame}[fragile]
\frametitle{Tuple Unpacking}
\begin{itemize}
\item Multiple variables can be assigned with values at one go using Tuples in Python.
\item This underlying implementation has a bigger advantage in Python in a few  operations such as swapping two/more variables.
\item In other programming languages, particularly statically-typed language, it is more complicated than how it's done in Python.
\item Without the use of a temporary variable, unlike in other languages, in Python, two variables can be swapped as follows:
\begin{lstlisting}
a,b = b,a
\end{lstlisting}
\end{itemize}
\end{frame}

\begin{frame}
\frametitle{More Examples in Tuples}
More code examples are available at --
 
\url{https://github.com/vivek14632/Python-Workshop/tree/master/Introducing\%20Python/Chapter\%203}
\end{frame}

\begin{frame}
\frametitle{Introducing Sets}
\begin{itemize}
\item A set is an unordered collection of zero or more immutable Python data objects. 
\item Sets do not allow duplicates and are written as comma-delimited values enclosed in curly braces. 
\item The empty set is represented by set(). \item Sets are \textit{heterogeneous} similar to lists and tuples.
\item Even though sets are not considered to be sequential, they do support a few of the familiar operations presented earlier in lists and tuples.
\end{itemize}
\end{frame}

\begin{frame}
\frametitle{Applications and Operations Using Sets}
\begin{itemize}
\item Common applications include membership testing, removing duplicates from a sequence, and computing mathematical operations such as intersection, union, difference, and symmetric difference.
\item Like other collections, sets support x in set (membership), len(set), and "for x in set" (iteration). 
\item Being an unordered collection, sets do not record element position or order of insertion. Accordingly, sets do not support indexing, slicing, or other sequence-like behavior.
\end{itemize}
\end{frame}

\begin{frame}
\frametitle{Frozen Sets}
\begin{itemize}
\item There are currently two builtin set types in Python, set and frozenset. 
\item The set type is mutable -- the contents can be changed using methods like add() and remove(). 
\item Since it is mutable, it has no hash value and cannot be used as either a dictionary key or as an element of another set. 
\item The frozenset type is immutable and hashable -- its contents cannot be altered after it is created; however, it can be used as a dictionary key or as an element of another set.
\end{itemize}
\end{frame}

\begin{frame}
\frametitle{More Examples in Sets}
More code examples are available at --
 
\url{https://github.com/vivek14632/Python-Workshop/tree/master/Introducing\%20Python/Chapter\%203}
\end{frame}


\begin{frame}
\frametitle{Exercises}
Create a list using list comprehension whose elements are the cubes of the first 10 natural numbers.

\end{frame}

\begin{frame}
\frametitle{Summary}
\begin{itemize}
\item We understood Sequences in Python.
\item We extensively experimented with Python's sequential data types such as Lists, Tuples and sets.
\item We learned the difference between the different sequence types and how they help us in different situations.
\item We understood and worked with the different methods that the sequence types provide.
\end{itemize}
\end{frame}
\end{document}
