\documentclass{beamer}
\usepackage[utf8]{inputenc}
\usepackage{graphicx}
\usepackage{subcaption}
\usepackage{listings}
%\usepackage{subfig}
\usetheme{Copenhagen}
\usecolortheme{seahorse}
 
 
%Information to be included in the title page:
\title{Python Sequences}
\author{Vivek K. S., Deepak G.}
\institute{Information Systems Decision Sciences (ISDS)\\
MUMA College of Business\\
University of South Florida \\
Tampa, Florida}
\date{2017}
 
\begin{document}
\frame{\titlepage}
\begin{frame}
\frametitle{Objectives}
\begin{itemize}
\item To understand built-in data sequence types such as Lists, Sets, Tuples and Dictionaries.
\item To understand their use and advantages.
\item To understand iterations and constructs that support sequences.
\item To learn the different operations that could be performed on these types.
\end{itemize}
\end{frame}
\begin{frame}
\frametitle{Lists}
\begin{itemize}
\item Lists are a collection of data elements arranged sequentially.
\item They are mutable in nature, in that they can be appended to, and data removed from.
\item Lists are enclosed by a pair of squared brackets '[]'
\item Lists in Python are heterogeneous. A list can contain any different type of data.
\item Lists are especially good for keeping track of things that might change.
\item A list could be created by saying [] or list().
\end{itemize}
\end{frame}


\begin{frame}[fragile]
\frametitle{Creating a List of Lists}
A list could be created with other lists as elements of it.
\begin{lstlisting}[language=Python]
>>>cities = ['Chicago','San Francisco','New York']
>>>metros = ['Dallas','Las Vegas', cities]
['8','11','2017']
\end{lstlisting}
\begin{itemize}
\item Changing the cities list here modifies the  metros list as well.
\item This is because, cities is just a reference to the block of memory containing "cities list" and that is being reflected in the metros list.
\end{itemize}
\end{frame}

\begin{frame}[fragile]
\frametitle{Data Type Conversions in Lists}
Converting a string to a list as easy as list('Python')

A string could also be converted into a List by using the split() method by splitting at a character of our choice.

\begin{lstlisting}[language=Python]
>>>birthday = '8/11/2017'
>>>birthday.split('/')
['8','11','2017']
\end{lstlisting}
A list can be converted to a String using join() as follows:
\begin{lstlisting}
>>>list = ['I','love','Python']
>>>' '.join(list)
'I love Python'
\end{lstlisting}
\end{frame}

\begin{frame}
\frametitle{Common List Operations}
The following are the common operations in List.
\begin{itemize}
\item Adding elements to the list using add, append and insert.
\item Indexing and Slicing.
\item Modifying the list using indexing and Slicing techniques.
\item Concatenating Lists.
\item Removing items from the list using "del" remove and pop.
\item Determining the membership of an item and counting the number of items.
\item Searching and Sorting the list.

\end{itemize}
\end{frame}

\begin{frame}
\frametitle{Tuples}
Similar to lists, tuples are sequences of arbitrary items. 
\begin{itemize}
\item Tuples are is the immutable collection sequence type in Python.
\item Tuples can’t be modified using add, delete, or changing of items after the tuple is defined. A tuple is this a constant list.
\item A tuple can be created using ().
\item Notice that a list can be created as follows "list()".
\end{itemize}
\end{frame}

\begin{frame}
\frametitle{Tuple Unpacking}
\begin{itemize}
\item Multiple variables can be assigned at one go using Tuples in Python
\item This underlying nature has a bigger advantage in Python is a few  operations such as variable swapping.
\item In other programming languages, particularly statically-typed language, it is more complicated than how it's done in Python.
\item Without the use of a temporary variable, unlike in other languages, in Python, two variables can be swapped as follows:
a,b = b,a
\end{itemize}
\end{frame}

\begin{frame}
\frametitle{Summary}
\begin{itemize}
\item We understood what sequences are in Python.
\item We extensively experimented with Python's sequential data types such as Lists and Tuples.
\item We learned the difference between these two sequence types and how they help us in different situations.
\item We understood and worked with the different methods that these two structures provide.
\end{itemize}
\end{frame}
\end{document}
