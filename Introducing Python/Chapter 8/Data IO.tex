\documentclass{beamer}
\usepackage[utf8]{inputenc}
\usepackage{graphicx}
\usepackage{subcaption}
\usepackage{listings}
%\usepackage{subfig}
\usetheme{Copenhagen}
\usecolortheme{seahorse}
 

%Information to be included in the title page:
\title{Data I/O}
\author{Vivek K. S., Deepak G.}
\institute{Information Systems Decision Sciences (ISDS)\\
MUMA College of Business\\
University of South Florida \\
Tampa, Florida}
\date{2017}
 
\expandafter\def\expandafter\insertshorttitle\expandafter{%
\insertshorttitle\hfill%
\insertframenumber\,/\,\inserttotalframenumber}

\lstset{language=python,
		showstringspaces=false,
                basicstyle=\ttfamily,
                keywordstyle=\color{blue}\ttfamily,
                stringstyle=\color{red}\ttfamily,
                commentstyle=\color{purple}\ttfamily,
                morecomment=[l][\color{magenta}]{\#}
}


\begin{document}
\frame{\titlepage}

\begin{frame}
\frametitle{Memory Management}
\begin{itemize}
\item There are two types of memory, the RAM memory and Disk memories.
\item RAM is fast, expensive and volatile.
\item Disk can hold more data, slower than RAM, cheap and non-volatile.
\item Programming requires persistence, that is storing and retrieving of data using non-volatile disks such as disks.
\end{itemize}
\end{frame}

\begin{frame}
\frametitle{File I/O}
\begin{itemize}
\item The simplest way to persist data is to use a flat file. This is just a sequence of bytes stored under a filename.
\item Python's file operations are modeled on the familiar and popular Unix equivalents and it helps us to read from/write into a file in memory.
\item The general syntax is 
fileobj = open(filename, mode)
\item Here, mode is a string indicating the file’s type and what you want to do with it.

\end{itemize}
\end{frame}

\begin{frame}
\frametitle{Mode}
The first letter of mode indicates the operation
\begin{itemize}
\item r means read.
\item w means write. If the file doesn't exist, its created. If the file does exist, its overwritten.
\item x means write, but only if the file does not already exist.
\item a means append (write after the end) if the file exists.
\end{itemize}
The second letter of mode is the file’s type:
\begin{itemize}
\item t (or nothing) means text.
\item b means binary.
\end{itemize}
\end{frame}

\begin{frame}[fragile]
\frametitle{Writing a Flat file}
\begin{lstlisting}

review = "I have a few leave Bluetooth and \
I'm gonna say this really brings the BOOM.\
My music is bumping the colors are amazing \
and it also plays FM radio how amazing is that?!"

review_file = open('review','wt')
review_file.write(review)
review_file.close()
\end{lstlisting}

The write() function returns the number of bytes written. It does not add any spaces
or newlines, as print() does.
\end{frame}


\begin{frame}[fragile]
\frametitle{Writing Large files in Chunks}
If we have a large source string, we can also write chunks until the source is done:

\begin{lstlisting}
def write_in_chunks(text,chunk):
    fout = open('output_file', 'wt')
    size = len(text)
    offset = 0
    chunk = chunk
    while True:
        if offset > size:
            break
        fout.write(text[offset:offset+chunk])
        offset += chunk
    fout.close()
\end{lstlisting}
\end{frame}


\begin{frame}[fragile]
\frametitle{Preventing Overwriting of Important Data}
If we want to prevent overwriting of our data, we can prevent it with "xt". Let's implement that with a try..except block.
\begin{lstlisting}

try:
	fout = open('output_file','xt')
	fout.write('Lets overwrite your data')
except FileExistsError:
	print('The file already exists!. 
	That was a close one.')
	
'The file already exists!. 
	That was a close one.'
\end{lstlisting}
\end{frame}


\begin{frame}[fragile]
\frametitle{Read a file using Read() in Chunks}
We can provide a maximum character count to limit how much read() returns at one
time because read() method reads at one go and that could strain the memory.

\begin{lstlisting}
review = ''
fin = open('output_file', 'rt')
chunk = 100
while True:
    fragment = fin.read(chunk)
    if not fragment:
        break
    review += fragment
fin.close()
len(review)
\end{lstlisting}
\end{frame}

\begin{frame}[fragile]
\frametitle{Reading Using readline()}
We can also read one line at a time using readline()
\begin{lstlisting}
review = ''
fin = open('review', 'rt' )
while True:
    line = fin.readline()
    if not line:
        break
    review += line

fin.close()
\end{lstlisting}
\end{frame}

\begin{frame}[fragile]
\frametitle{Reading Using Iterator}
The easiest way is perhaps to use an iterator.
\begin{lstlisting}
review = ''
fin = open('review', 'rt' )
for line in fin:
    print(line)
    review += line

fin.close()
\end{lstlisting}
\end{frame}

\begin{frame}[fragile]
\frametitle{Closing Files with With}

\begin{itemize}
\item In Python, if we forget to close a file that we have opened, it will be automatically closed if the file operation is happening as part of a function call.
\item In that case, the file is closed when the function execution stops.
\item If otherwise, the file has to be closed explicitly, and in this case we can use "with" to accomplish automatic closing of the file.
\item Python has context managers to clean up things such as open files.
\end{itemize}
\begin{lstlisting}
with open('review', 'wt') as fout:
    fout.write(poem)
\end{lstlisting}
\end{frame}


\begin{frame}[fragile]
\frametitle{Seek and Tell}
\begin{itemize}
\item Python keeps track of where we are whenever we read and write in file operations.
\item The tell() returns the current offset from the beginning of the file  in bytes.
\item Seek() lets us jump to another byte offset in the file.

\begin{lstlisting}
fin = open('review', 'rt')
fin.tell()
\end{lstlisting}
\item Lets use seek to read some particular byte:
\begin{lstlisting}
fin = open('review', 'rt')
fin.seek(50) # reading from the byte 50
\end{lstlisting}
\end{itemize}
\end{frame}

\begin{frame}
\frametitle{Complete Coding Exercise}
Complete coding exercise is available at \url{https://github.com/vivek14632/Python-Workshop/tree/master/Introducing\%20Python/Chapter\%206}
\end{frame}

\begin{frame}
\frametitle{Summary}
\begin{itemize}
\item We learned how flat files can be used to create and store quick data.
\item We learned how to perform different file operations such as read, write and use of iterators to read and write data.
\item We learned the use of 'with' for automatically close files that may accidentally be left open.
\item We learned how to prevent overwriting of critical data using exception mechanism.
\end{itemize}
\end{frame}

\begin{frame}

\frametitle{Exercise}

Read tweets data from 1.json file and extract different attributes from the tweets such as text, id, date, etc.

\end{frame}

\end{document}
