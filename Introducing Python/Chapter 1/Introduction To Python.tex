\documentclass{beamer}
\usepackage[utf8]{inputenc}
\usepackage{graphicx}
\usepackage{subcaption}
\usepackage{listings}
%\usepackage{subfig}

\usetheme{Madrid}
\usecolortheme{seahorse}
\usefonttheme{serif} 
 
 
%Information to be included in the title page:
\title{Introduction to Python}
\author{Vivek K. S., Deepak G.}
\institute{Information Systems Decision Sciences (ISDS)\\
MUMA College of Business\\
University of South Florida \\
Tampa, Florida}
\date{2017}
 
\expandafter\def\expandafter\insertshorttitle\expandafter{%
  \insertshorttitle\hfill%
  \insertframenumber\,/\,\inserttotalframenumber}
  

\lstset{language=python,
		showstringspaces=false,
                basicstyle=\ttfamily,
                keywordstyle=\color{blue}\ttfamily,
                stringstyle=\color{red}\ttfamily,
                commentstyle=\color{green}\ttfamily,
                morecomment=[l][\color{magenta}]{\#}
}

 
\begin{document}
 
\frame{\titlepage}

\begin{frame}
\frametitle{Bootcamp Outline}
The following topics will be discussed in these four days.
\begin{itemize}
\item Introduction to Python Data Types.
\item Object-oriented Programming in Python.
\item File and Data I/O.
\item Scientific computing using Numpy. 
\item Pandas for Data Analysis.
\item Data Visualization using Matlplotlib.
\item Data Structures and Algorithms using Python.
\item Abstract Data Structures using Python.
\item Searching and sorting algorithms.
\item Tree Data Structures using Python.
\item Opportunities in Python. 
\end{itemize}
\end{frame}


\begin{frame}
\frametitle{Introduction to Python}
\begin{itemize}
\item Python is a general-purpose, open-source, high-level, dynamically typed, and interpreted language.
\item It has a very easy to understand syntax and easy prototyping ability.
\item It has a gentle learning curve that helps a  programmer to be productive at a very early stage of learning.
\item It is relatively terse compared to other languages and requires comparatively  a few lines of code that could take more number of lines of code in other languages to solve a similar problem.
\item Python is currently being used in multiple areas of computer science such as web development, machine learning, Neural networks and also in Quantum computing.
\end{itemize}
\end{frame}

\begin{frame}
\frametitle{Python versus Other Languages}
\begin{itemize}
\item Python is generally slower compared to C, C++ for computationally intensive applications.
\item Where it lacks in speed relatively, it gains in speed of development which helps a programmer experiment more.
\item The standard Python interpreter is implemented in C and this implementation is called "CPython".
\item Python interpreters are becoming faster and newer implementations such as "PyPy" are faster than the CPython implementation.
\end{itemize}
\end{frame}

\begin{frame}
\frametitle{Python 2 versus Python 3}
\begin{itemize}
\item Python 2 has been around for so long and comes as part of Linux and Apple machines.
\item Python 3 is an improvement over Python 2, that has overcome many drawbacks of the language and is currently being widely adopted.
\item The last version of Python 2, 2.7 is still supported and will be in general usage. However, it is the last of the series.
\item Some of the most prominent changes seen in Python 3 are the print statement, string formatting, and use of the Unicode Standard for text data.
\end{itemize}
\end{frame}

\begin{frame}
\frametitle{Let's write Our First Python Program}
Now that we have learned what Python is about, let's go ahead and write our first Python program.

\begin{itemize}
\item Open a file in your machine using either using notepad or a text editor of your choice.
\item Type "print("Hello World!!")" and save the file.
\item Name the file as \textit{first\_program.py} and save it.
\item Open the terminal/command prompt and navigate to the path where the \textit{.py} file has been saved.
\item Type the following command:

python file\_name.py and hit enter and wait for the magic to happen.
\item We have performed the first ritual of saying "Hello World!!" to our fellow programmers.
\end{itemize}
\end{frame}



\begin{frame}
\frametitle{Anaconda and Jupyter}
We will be practicing all the exercises on Jupyter Notebooks and we recommend that you have an Anaconda distribution installed on your machine before we go ahead. Here are a few useful links.
\begin{itemize}
\item Download anaconda at \url{https://www.continuum.io/downloads}
\item Anaconda installation instructions \url{https://docs.continuum.io/anaconda/install/windows}
\item The material for the discussion is available at \url{https://github.com/vivek14632/Python-Workshop}
\end{itemize}
\end{frame}

\begin{frame}
\frametitle{How to start Jupyter}
\begin{itemize}
\item Start Anaconda Prompt on your machine.
\item Change directory to the folder containing the files downloaded from GitHub repository.
\item Once, you are in the correct directory, type  "jupyter notebook".
\item The editor should open on your browser at \url{http://localhost:8888/tree}
\item Click new and choose Python (under Notebooks). You will see a new notebook open in a new window.
\item Type the following line in the first cell "print('Hello World!!')" and click on the play button (or) Cell and choose run cells.
\item You should see the message "Hello World!!" get printed right beneath the cell.
\item You have written your first Python code on Jupyter Notebook now.
\end{itemize}
\end{frame}
\begin{frame}
\frametitle{We Begin Now}
\centering
Hope you enjoyed the Introduction. There is more to come and we hope you all enjoy 4 days of learning with us.


Thank you. 
\end{frame}
\end{document}

