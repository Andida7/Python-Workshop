\documentclass{beamer}
\usepackage[utf8]{inputenc}
\usepackage{graphicx}
\usepackage{subcaption}
\usepackage{listings}
%\usepackage{subfig}
\usetheme{Copenhagen}
\usecolortheme{seahorse}
 
 
%Information to be included in the title page:
\title{Code Structures in Python}
\author{Vivek K. S., Deepak G.}
\institute{Information Systems Decision Sciences (ISDS)\\
MUMA College of Business\\
University of South Florida \\
Tampa, Florida}
\date{2017}
 
\begin{document}
\frame{\titlepage}

\begin{frame}
\frametitle{Objectives}
\begin{itemize}
\item To understand coding constructs like conditional statements, loops and iterations.
\item To understand Comprehensions using Sequential Data Types.
\item To Learn and implement Functions and Generators.
\item To understand exceptions and ways to solve them. 
\end{itemize}
\end{frame}

\begin{frame}[fragile]
\frametitle{If.. else Constructs}
Python, like most other programming languages implements conditional statements using if..else constructs.

In general the if..else statement is used to check if a condition is True or False and based on the decision, decide whether or not to go ahead with the subsequent instruction.
For example,
\begin{lstlisting}[language=Python]

interesting = True
if interesting:
	print('Learn more about it..!')
else:
	print('Move on..')
\end{lstlisting}
\end{frame}

\begin{frame}[fragile]
\frametitle{Checking Multiple Conditions}
More than one condition could be checked as follows:
\begin{lstlisting}[language=Python]

interesting = True
easy = True
if interesting and easy:
	print('Learn more about it..!')
else:
	print('Move on..')
\end{lstlisting}
\end{frame}

\begin{frame}[fragile]
\frametitle{Nested Conditions}
Conditional statements could be nested as follows:
\begin{lstlisting}[language=Python]

interesting = True
easy = True
if interesting:
	if easy:
		print('Easy for you..!')
	else:
		print('Keep practicing..!')
else:
	print('Move on..')
\end{lstlisting}
\end{frame}

\begin{frame}[fragile]
\frametitle{Multiple Conditional Possibilities}
In the event that there are more than 2 possibilities, we use a special construct called the elif which is just the shortened version of else..if.
\begin{lstlisting}[language=Python]

time = 'morning'
if time == 'morning':
    print('Its time for Breakfast!!')
elif time == 'noon':
    print('Its time for lunch')
elif time == 'night':
    print('Its time for lunch')
else:
    print('I am hungry all the time anyways :(')
	
\end{lstlisting}
\end{frame}

\begin{frame}[fragile]
\frametitle{Coding Exercise}
Try and answer why only the first if loop is getting executed and not the second one.
\begin{lstlisting}[language=Python]
a = [1,2,3]
b = [1,2,3]

if a==b:
    print('They are equal')
if a is b:
    print('I said they are equal!!')
	
\end{lstlisting}
\end{frame}

\begin{frame}[fragile]
\frametitle{Coding Exercise}
Now that you have mastered the art of variable comparison in Python, answer this.
\begin{lstlisting}[language=Python]
a = 256
b = 256
a == b
True
a is b
True

a = 257
b = 257
a == b
True
a is b
False
	
\end{lstlisting}
\end{frame}

\begin{frame}[fragile]
\frametitle{Possible False Conditions}
These are some of the possible conditions that could result in False
\begin{lstlisting}[language=Python]
boolean False
null None
zero integer 0
zero float 0.0
empty string ''
empty list []
empty tuple ()
empty dict {}
empty set set()
\end{lstlisting}
\end{frame}

\begin{frame}[fragile]
\frametitle{Examples}
\begin{lstlisting}[language=Python]
# Example 1
a = 0
if a:
	print('Its  True')
else:
	print('Its False')

# Example 2
def add(x,y):
    if type(x) is int and type(y) is int:
        return(x+y)

if sum:
    print('The sum is', sum)
else:
    print('I dont see no sum')
\end{lstlisting}
\end{frame}

\begin{frame}[fragile]
\frametitle{Repeat with While}
While is the simplest looping construct that helps us to repeat a step any number of times as long as the condition evaluates to True.
\begin{lstlisting}[language=Python]
count = 1
while count < 5:
    print('The count is', count)
    count += 1
\end{lstlisting}
Using break can help us break the loop midway.
\begin{lstlisting}[language=Python]
count = 1
while count < 5:
    print('The count is', count)
    count += 1
    if count == 3:
    	break
\end{lstlisting}
\end{frame}

\begin{frame}[fragile]
\frametitle{Use of Continue}
Continue can be used to continue with the loop.
\begin{lstlisting}[language=Python]
while True:
    value = input("Enter an integer of choice. Press q to quit: ")
    if value == 'q': # quit
        break
    else:
        number = int(value)
        if number%2 == 0:
            print(number, "squared is",\
             number*number)
            continue
        else:
            break
        
\end{lstlisting}
\end{frame}

\begin{frame}
\frametitle{Summary}
\begin{itemize}
\item We understood looping and conditional structures in Python.
\item We learned the need for iteration and how it makes programming of repeated tasks easier.
\item We understand looping structures such as if..else, while and for and other keywords such as break and continue.
\item We also learned nesting and other complex code structures.
\end{itemize}
\end{frame}
\end{document}