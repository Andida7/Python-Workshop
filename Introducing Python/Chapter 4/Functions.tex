\documentclass{beamer}
\usepackage[utf8]{inputenc}
\usepackage{graphicx}
\usepackage{subcaption}
\usepackage{listings}
%\usepackage{subfig}
\usetheme{Copenhagen}
\usecolortheme{seahorse}
 

%Information to be included in the title page:
\title{Functions in Python}
\author{Vivek K. S., Deepak G.}
\institute{Information Systems Decision Sciences (ISDS)\\
MUMA College of Business\\
University of South Florida \\
Tampa, Florida}
\date{2017}

\expandafter\def\expandafter\insertshorttitle\expandafter{%
\insertshorttitle\hfill%
\insertframenumber\,/\,\inserttotalframenumber}

\lstset{language=python,
		showstringspaces=false,
                basicstyle=\ttfamily,
                keywordstyle=\color{blue}\ttfamily,
                stringstyle=\color{red}\ttfamily,
                commentstyle=\color{purple}\ttfamily,
                morecomment=[l][\color{magenta}]{\#}
}

 
\begin{document}
\frame{\titlepage}

\begin{frame}
\frametitle{Functions in Python}
\begin{itemize}
\item Functions are the best way to write reusable code.
\item Simply put, its a block of code that may/may not take input(s) and performs a specific operation over it/them.
\item A function may or may not return a value.
\item Functions improve re-usability and readability.
\item Functions have a name and that's how they are referred to by the calling object in the program.
\item Functions need to be defined before being called.
\item In Python, we define a function using the "def" keyword.
\end{itemize}
\end{frame}

\begin{frame}[fragile]
\frametitle{A simple Function in Python}
The following function shows a typical Python function.
\begin{lstlisting}[language=Python]
	def add(x,y):
		return(x+y)
\end{lstlisting}
\begin{itemize}
\item Its defined using "def".
\item It has a name "add" and takes in two arguments x and y.
\item Notice that x and y are not limited by any data type and are mere references that point to an object in memory.
\item The function returns the sum of the two variables.
\item Notice that there is no need to use curly braces or enclosing mechanism to enclose the function body.
\item Instead, Python uses indentation to mark a block of code.
\end{itemize}

\end{frame}

\begin{frame}[fragile]
\frametitle{Continued..}
In the previous example, the function returned a value. In some situations, there could be functions that don't return anything at all.
\begin{lstlisting}[language=Python]
def print_something(message):
	print(message)
\end{lstlisting}

The following function is a special case where there is no return, neither does it do anything. We use a special keyword 'pass' in Python to accomplish this.
\begin{lstlisting}[language=Python]
def do_nothing():
	pass
\end{lstlisting}
\end{frame}

\begin{frame}[fragile]
\frametitle{Enclosing Code Blocks in Function}
A real function that does an actual operation looks like this:
\begin{lstlisting}[language=Python]
import datetime
def meal_time(time):
    print(time)
    if time<=10 and time>= 8:
        print('Time for Breakfast')
    if time<=14 and time>= 12:
        print('Time for Lunch')
    if time<=20 and time>= 18:
        print('Time for Dinner')

hour = datetime.datetime.now().hour
meal_time(hour)
\end{lstlisting}

\end{frame}

\begin{frame}[fragile]
\frametitle{Positional Arguments}
\begin{itemize}
\item Python handles function arguments in an exceptionally flexible manner compared to other languages.
\item Having Positional arguments is the most common approach to passing parameters to the function in order.
Example,
\begin{lstlisting}[language=Python]
def language_rating(lang, rating):
	print('I rate {} {}/10.'.\
	format(lang,rating))
\end{lstlisting}
\item The problem with Positional Arguments is that we need to remember the order of the arguments while passing values to them.
\end{itemize}
\end{frame}

\begin{frame}[fragile]
\frametitle{Keyword Arguments}
To avoid the confusion of Positional arguments, we have keyword arguments.
\begin{itemize}
\item Here, we can specify arguments by the names of their corresponding parameters.
\item This avoids the confusion of ordering.
Example,
\begin{lstlisting}[language=Python]
def language_rating(lang, rating):
	print('I rate {} {}/10.'.\
	format(lang,rating))

language_rating(rating=10,lang='Python')
\end{lstlisting}
\end{itemize}
\end{frame}



\begin{frame}[fragile]
\frametitle{Mixing Positional and Keyword Arguments}
Keyword and Positional arguments could be mixed as well. But it should be noted that Positional arguments should always be specified first.
\begin{lstlisting}[language=Python]
def language_rating(lang, rating):
	print('I rate {} {}/10.'.\
	format(lang,rating))

language_rating('Python',rating=10)
\end{lstlisting}
\end{frame}

\begin{frame}[fragile]
\frametitle{Specifying Default Parameter Values}
Arguments can be set with default values as follows:

\begin{lstlisting}
def language_rating(rating, lang='Python'):
	print('I rate {} {}/10.'.\
	format(lang,rating))

language_rating(10)
\end{lstlisting}
Note that the default arguments should come in the end.
Passing values to the default argument will replace the default value as follows:
\begin{lstlisting}
def language_rating(rating, lang='Python'):
	print('I rate {} {}/10.'.\
	format(lang,rating))

language_rating(10,'Java')
\end{lstlisting}
\end{frame}


\begin{frame}[fragile]
\frametitle{Positional Arguments with *}
The asterisk can be used to accept variable number of arguments.
\begin{itemize}
\item The asterisk groups a variable number of positional arguments into a tuple of parameter values.
\end{itemize}
\begin{lstlisting}
def print_args(*args):
    print('Positional argument tuple:', args)

print_args('hello','world',5,6.0,True)

# Calling the method with no arguments:
print_args()
Positional argument tuple: ()
\end{lstlisting}
\end{frame}

\begin{frame}[fragile]
\frametitle{Keyword Arguments with **}
We could use "**" to gather keyword arguments into a dictionary where the argument names are the keys and values are the corresponding dictionary values:

\begin{lstlisting}
def print_kwargs(**kwargs):
    print('Keyword arguments:', kwargs)
print_kwargs(NYC='Albany',
 California='Sacramento')
Keyword arguments: {'NYC': 'Albany',
 'California': 'Sacramento'}
\end{lstlisting}
\end{frame}

\begin{frame}[fragile]
\frametitle{Passing Functions as Arguments}
We can pass Functions as arguments to other functions as functions are objects in Python.
\begin{lstlisting}[language=Python]
	def add(x,y):
		return(x+y)
	def arithmetic(func,x,y):
		return(func(x,y))
	arithmetic(add,5,10)
	15
\end{lstlisting}
\end{frame}

\begin{frame}[fragile]
\frametitle{Contd..}
Another interesting example.
\begin{lstlisting}[language=Python]
def edit_text(sentence, func):
    words = sentence.split()
    final_sentence = " ".join(func(word)\
     for word in words)
    return(final_sentence.strip())  
def capitalize_sentence(word):
    return word.capitalize()
    
sentence = 'python was authored by 
Guido van Rossum'
edit_text(sentence,capitalize_sentence)
'Python Was Authored By Guido Van Rossum'
\end{lstlisting}
\end{frame}

\begin{frame}[fragile]
\frametitle{Lambda Functions}
Lambda functions are used to create anonymous "on the go" functions in place of little function definitions that might not be reused later and is an overkill.

Lambda functions can be expressed in a single statement.
\begin{lstlisting}[language=Python]
my_lambda = lambda x : x*2

print(my_lambda(10))

\end{lstlisting}
\end{frame}

\begin{frame}
\frametitle{Complete Coding Exercise}
Complete Coding Exercise is available at --
 
\url{https://github.com/vivek14632/Python-Workshop/tree/master/Introducing\%20Python/Chapter\%204}
\end{frame}


\begin{frame}
\frametitle{Summary}
\begin{itemize}
\item We learned functions  in Python and how they help us to enclose blocks of code that perform a defined functionality, thereby improving reusability and readability.
\item We learned how to modularize our program.
\item We learned how important functions are important to the Object-oriented approach of coding.
\item We learned about arguments and parameters and the different type of arguments.
\item We experimented with the different types of argument to tackle different requirements.
\end{itemize}
\end{frame}
\end{document}