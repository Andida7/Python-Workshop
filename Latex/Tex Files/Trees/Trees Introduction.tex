\documentclass{beamer}
\usepackage[utf8]{inputenc}
\usepackage{graphicx}
\usepackage{subcaption}
\usepackage{listings}
%\usepackage{subfig}

\usetheme{Madrid}
\usecolortheme{seahorse}
\usefonttheme{serif} 
 
 
%Information to be included in the title page:
\title{Tree Data Structure}
\author{Vivek K. S., Deepak G.}
\institute{Information Systems Decision Sciences (ISDS)\\
MUMA College of Business\\
University of South Florida \\
Tampa, Florida}
\date{2017}
 
 
 
\begin{document}
 
\frame{\titlepage}
 
\begin{frame}
\frametitle{Introduction to Trees}
\begin{itemize}


\item Trees are a different type of a data structure unlike the Linear data structures like Stacks and Queues.

\item Trees are used in many areas of computer science, including operating systems, graphics, database systems, and computer networking.

\item Conceptually a tree is made up of a root, branches, and leaves.

\item A document object model is an example for a tree data structure.
\end{itemize}

\end{frame}

\begin{frame}
\frametitle{An HTML Object Tree}
\end{frame}

\begin{frame}
\frametitle{Properties of a Tree}
The tree has the following properties.
\begin{itemize}
\item Each level of the tree represents a unique logical path in which the nodes are arranged in relation to one another.

\item For example, all the  information that pertains to the body of the html page comes under the html Node and not under the "Head" node.

\item Second, the children of one node are different from another node's children even if they have the same names and may contain the same data.

\item For example, in the html page, there could be muliplte ul ("unordered lists") elements each of them containing one or many li (List items) elements which are independent from one another.

\item Third, each node is unique.

\item For example, in the file system of your machine, each file and folder has a unique folder path that can be used to access that file or folder.
\end{itemize}
\end{frame}


\begin{frame}
\frametitle{Taxonomy of a Tree}
The following are some of the terms associated with a Tree.

\begin{itemize}
\item Node - The Node is the fundamental building block of a tree. It can have a name that we call the "key" and and an optional payload. The payload is the data contained by the node.

\item Edge = The edge connects two nodes to show the relationships between the nodes. Except the root node, every other node in a tree has exactly one incoming edge. 

\item Each node may have several outgoing edges. In a Binary tree, each node only has two outgoing nodes.
\item Root - It is the root of the tree and has no incoming edges.
\item Path - The ordered list of nodes that are connected by nodes.


\end{itemize}
\end{frame}

\begin{frame}
\frametitle{Continued..}

\begin{itemize}
\item Children - The nodes that have incoming edges from the same node.
\item Parent - The node that connects to all child nodes with edges.
\item Sibling -  All the nodes that share the same Parent node.
\item Subtree - A subset of the tree comprised of a parent and all the descendants of that parent.
\item Leaf Node - A node with no children of its own.
\item Level - The level of a node 'n' is the number of edges on the path from the root node to 'n'.
\item Height - The height of a tree is equal to the maximum level of any node in the tree.
\end{itemize}
\end{frame}

\begin{frame}
\frametitle{Essential functions in a Tree}
The Tree has the following essential functions. We are specifically going to discuss a binary tree.
\begin{itemize}

\item Ability to create new instances of a tree.
\item Ability to get the subtree corresponding to the child node of the current node.
\item Ability to create a new binary tree and install it as the child of the current node.
\item Ability to set and get the value of the root of the tree.

\end{itemize}
The key decision in implementing a tree is choosing an internal storage technique.
\end{frame}

\begin{frame}
\frametitle{Implementing a Tree in Python}
In Python there are two ways to build a Tree,

\begin{itemize}
\item List of lists representation.
\begin{itemize}
\item Each subtree is represented as a list of lists.
\item The list contains the value of the root as the first element and the left and right subtrees as lists themselves.
\item The nodes can be used by List indexing.
\end{itemize}
\item Nodes and references technique.
\begin{itemize}
\item We use an object-oriented paradigm with 
class representations for the root value, as well as the left and right subtrees.
\end{itemize}
\end{itemize}
 
\end{frame}

\begin{frame}[fragile]
\frametitle{Implementation in Python}
We will explore the second method "Nodes and References" more in detail.

We can create a simple definition of the node as follows.
\begin{lstlisting}[language=Python, keywordstyle=\color{blue}]

class BinaryTree:
	def __init__(self, root):
		self.key = root
		self.left_child = None
		self.right_child = None

\end{lstlisting}
In this definition, the constructor expects an object to be stored in the root value. We can store any type of data here. Lets store the character 'A' and the following alphabets in the child nodes so that it helps to improve our understanding of Trees.
\end{frame}

\begin{frame}[fragile]
\frametitle{Inserting a Left Child}
To add a left child, we could create an instance of the Binary Tree and set the left child attribute of the root to point to this new object.
Thus, it becomes a complete Binary Tree (subtree) of its own.

\begin{itemize}
\item When the left child does not exist, we create a new Binary Tree object and assign the value to it.
\item When the left child already exists, we simply insert a node and push the existing child down one level in the tree. 
\end{itemize}

\begin{lstlisting}[language=Python, keywordstyle=\color{blue}]

def insert_left(self,new_node):
        if self.left_child == None:
            self.left_child = BinaryTree(new_node)
        else:
            temp = BinaryTree(new_node)
            temp.left_child = self.left_child
            self.left_child = temp

\end{lstlisting}

\end{frame}


\begin{frame}[fragile]
\frametitle{Inserting a Right Child}
To add a right child, the method is the same.

\begin{lstlisting}[language=Python, keywordstyle=\color{blue}]

def insert_right(self,new_node):
        if self.right_child == None:
            self.right_child = BinaryTree(new_node)
        else:
            temp = BinaryTree(new_node)
            temp.right_child = self.right_child
            self.right_child = temp

\end{lstlisting}

\end{frame}

\begin{frame}[fragile]
\frametitle{Accessor Methods to get the Child Nodes}
\begin{lstlisting}[language=Python, keywordstyle=\color{blue}]
def get_right_child(self):
	return self.right_child
def get_left_child(self):
	return self.left_child
def set_root_val(self,obj):
	self.key = obj
def get_root_val(self):
	return self.key
\end{lstlisting}

\end{frame}
\end{document}