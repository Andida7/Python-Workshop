\documentclass{beamer}
\usepackage[utf8]{inputenc}
\usepackage{graphicx}
\usepackage{subcaption}
\usepackage{listings}
%\usepackage{subfig}

\usetheme{Copenhagen}
\usecolortheme{seahorse}
 
 
%Information to be included in the title page:
\title{Deque Data Structure}
\author{Vivek K. S., Deepak G.}
\institute{Information Systems Decision Sciences (ISDS)\\
MUMA College of Business\\
University of South Florida \\
Tampa, Florida}
\date{2017}
 
 
 
\begin{document}
 
\frame{\titlepage}
 
\begin{frame}
\frametitle{What is a Deque}
\begin{itemize}

\item A Deque is an ordered collection of items with a variation.

\item It has two ends called the front and the rear which are interchangeable unlike the queue data structure.

\item Data items can be added and removed from both the ends and thus it assumes the characteristics of both a stack and a queue. 

\item It could be thought of as a crossover between a stack and a queue.

\item It does not follow any ordering principle like LIFO or FIFO.


\end{itemize}
\end{frame}


\begin{frame}
\frametitle{Essential Operations in a Deque}
The Deque must have the ability to
\begin{itemize}

\item To add a new item either to the front or the rear end of the deque.

\item To add or remove an item from both the front or the rear end of the deque.

\item To check if the deque is empty.

\item To find the size of the deque.

\end{itemize}
\end{frame}

\begin{frame}
\frametitle{Logical Approach to Implementing a Deque}
\begin{itemize}

\item We need to be able to create new queue instances on the go. Hence we will take an object oriented approach towards building a Queue and defining its behavior.

\item The list data structure provides us with the methods to perform all these operations.

\item The ability to add and remove items from both the ends of the list is provided by the list indices.

\item To add an element to the front, we can simply do a traditional list insert.

\item To remove elements from the front, we use the traditional pop() method.

\item This implementation of adding and removing items from the front is O(n) operation.

\item This is because, we have to traverse all the way to the end (which is the front) to be able to do add and remove items from the “front”.


\end{itemize}
\end{frame}


\begin{frame}
\frametitle{Continued..}
\begin{itemize}

\item Likewise, the removal of item from the rear can be done by using the pop() method at the index position 0.

\item To add an element to the rear, we will be inserting the element to the index position 0.

\item This implementation of adding and removing items from the front is O(1) operation.

\item Just like in queues we will use the len() and comparator operation to check the size and for empty deques.

\end{itemize}
\end{frame}

\begin{frame}[fragile]
\frametitle{Python code}

Code Implementation in Python
\begin{lstlisting}[language=Python]
class Deque:
	def __init__(self):
		self.items = []
	def is_empty(self):
		return self.items == []
	def add_to_front(self, item):
		self.items.append(item)
	def add_to_rear(self, item):
		self.items.insert(0,item)
	def remove_from_front(self):
		return self.items.pop()
	def remove_from_rear(self):
		return self.items.pop(0)
	def length(self):
		return len(self.items)
\end{lstlisting}

\end{frame}
\end{document}

