\documentclass{beamer}
\usepackage[utf8]{inputenc}
\usepackage{graphicx}
\usepackage{subcaption}
\usepackage{listings}
%\usepackage{subfig}

\usetheme{Copenhagen}
\usecolortheme{seahorse}
 
 
%Information to be included in the title page:
\title{Queue Data Structure}
\author{Vivek K. S., Deepak G.}
\institute{Information Systems Decision Sciences (ISDS)\\
MUMA College of Business\\
University of South Florida \\
Tampa, Florida}
\date{2017}
 
 
 
\begin{document}
 
\frame{\titlepage}
 
\begin{frame}
\frametitle{What is a Queue}
\begin{itemize}

\item A queue is an ordered collection of items.

\item Data items are added through one end called the “rear” and removed through the other called the “front”.


\item It follows the FIFO ordering principle also known as “first-come first-served.”
 

\item Ticket counters and Printing tasks in a library are examples of real world examples of a queue.



 

\end{itemize}
\end{frame}


\begin{frame}
\frametitle{Essential Queue Operations}

\begin{itemize}

\item Queue - creates a new queue that is empty. It needs no parameters and returns an empty.

\item enqueue - adds a new item to the rear of the queue. It needs the item and returns
nothing.

\item dequeue - removes the front item from the queue. It needs no parameters and returns the
item. The queue is modified.

\item empty checker - tests to see whether the queue is empty. It needs no parameters and returns a
boolean value.

\item size -  returns the number of items in the queue. It needs no parameters and returns an
integer.


\end{itemize}

\end{frame} 


\begin{frame}[fragile]
\frametitle{Python code}

This is the code for queue.
\begin{lstlisting}[language=Python]
import numpy as np
x=10
for i in range():
	print x
	
\end{lstlisting}

\end{frame}
\end{document}

